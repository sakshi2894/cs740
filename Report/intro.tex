
\section{Introduction}
\label{sec:intro}

Today, a humongous amount of data goes through various access networks, enterprise networks, data centers, cloud computing environments. These network systems require data to go through multiple network/service functions in a specific order. This gives rise to the notion of Service Function Chaining. On the other hand, many data centers have started adopting SDN based architecture. 

Being controlled by a centralized controller, SDNs open a completely new possibility of designing routing algorithms for service function chaining. There can be multiple deployments of a particular network function, hence there can be multiple paths chosen by a single flow depending upon the available resources. While directing traffic flow through these networks, we can maximize the throughput by choosing the optimal path while making sure that the service function chain requirement is satisfied. 

The complexity of this problem can be demonstrated with the network depicted in Figure~\ref{fig:nw_example}. This network contains four types of VNFs and each of them has multiple instances deployed. For example, VNF11, VNF12 and VNF13 indicate the first, second and third instances of VNF1 respectively. Now suppose that there's a flow with SFC request which starts from node A and has to traverse the instances of VNF1, VNF2, VNF3 and VNF4 before arriving at node J. As we can see in this network, there exists many paths (such as the dotted lines marked with different colors) that traverse different VNF instances and can satisfy the requirement. Therefore, the challenge for dynamic SFC formation is to make an optimal strategy selecting VNF instances from multi-instance NFV environment and routing flows with SFC requests to traverse these selected VNF instances in predefined orders.

\fig{width=\columnwidth}{VNF}{\textmd{A sample network with function nodes and service chaining specifications ~\cite{ref:paper2}}}{fig:nw_example}

This motivates us to study different traffic routing algorithms being proposed by various researchers, implement them and do a comparative study to determine the algorithm that works best in terms of throughput in various scenarios.

The traffic routing algorithms can broadly be divided into two classes - offline and online. In the case of offline routing algorithms it can be assumed that all required traffic demands are available a prior whereas in the case of online algorithms, the decisions have to be made using only the current state of network and the incoming flow. Since, the offline algorithms have more information about sets of network flows they generally result in a better network allocation. The downside however is that they take more time which can violate QoS requirements for flows and thus their use in real-time scenarios is limited.

In this paper we studied the offline algorithm Traffic-Merging Algorithm (TMA) proposed in ~\cite{ref:paper1}. We implemented TMA with goal of maximising throughput while putting constraints on link capacity, processing capacity usage of a middle box and number of routing rules installed for this flow. This utility problem as it turns out is NP hard. TMA, in the first phase, solves the optimization problem by formulating it as a constrained linear optimization problem and ignoring the non-convex constraint on number of rules that can be installed on a switch. TMA in the second phase then addresses the constraint on rules installed on switches by using a merge operator. This merge operator merges the routing rules obtained from the first phase and guarantees an upper bound on number of rules that need to be installed on switches without compromising on the network throughput. Since the merge operator does not change the network throughput obtained from linear optimization problem, as part of this paper, we only use the output from the first phase of this algorithm which we call Max Throughput Routing Algorithm or MTRA (the linear optimization problem) for our throughput calculation. 


For analysing algorithms in online environment, we compared Primal-Dual-Update Algorithm (PDA) proposed in ~\cite{ref:paper1} and RA-RA proposed in ~\cite{ref:paper2}. PDA accepts or rejects a new flow by keeping parsimonious price variables for each resource. PDA has a system parameter which allows a trade off in algorithm competitiveness in maximizing throughput and the routing scheme performance in meeting delay requirements. The values of this parameter is in the range of 0 to 1. This algorithm approaches offline optimum as this parameter approaches 0. RA-RA is a Differentiated Routing Problem considering SFC(DRP-SFC). It is formulated as a Binary Integer Programming(BIP) model. RA-RA solves DRP-SFC to obtain the routing that maximises the throughput of the network by ensuring efficient consumption of resources.


The rest of the paper is organised as follows: Section~\ref{sec:problemdef} gives an overview of problem formulation for the three algorithms in consideration, Section~\ref{sec:setup} talks about our experimental setup for the comparison of these algorithms and in Section~\ref{sec:results} we evaluate the results. We also discuss the related works in the field in Section~\ref{sec:relatedwork}. Finally, we conclude and discuss future work in Section~\ref{sec:conclusion}.


