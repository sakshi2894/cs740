\section{Results}
\label{sec:results}

TMA, PDA and RA-RA were run for different number of flows and for different topologies. The simulation was conducted on Fat Tree and CORONET CONUS topologies. The number of flows was varied from 100 to 1000 for PDA and RA-RA. For TMA, the upper bound for the number of flows was limited to 400 in the case of CORONET CONUS and to 300 in the case of Fat-Tree due to computation restrictions.  The number of pods in Fat Tree was set to 8. We measured the throughput for different number of flows. The number of flows vs throughput was plotted as shown in Figure ~\ref{fig:graph}. Figure ~\ref{fig:sub1} and Figure ~\ref{fig:sub3} compares the online algorithms PDA and RA-RA. We consider PDA under two settings : (1) $\epsilon = 0.5$ (2) $\epsilon = 0.75$. Figure ~\ref{fig:sub2} and ~\ref{fig:sub4} compares TMA with the upper bound.

The Upper Bound for throughput is the maximum achievable throughput across all flows. This is the ideal case that would be achieved when all the flows' SFC request is satisfied, which indicates that each flow in the group gets its desired bandwidth and CPU share. It is assumed that the upper bound is obtained in an environment without any constraints.

According to \cite{ref:paper1}, TMA achieves the maximum throughput under practical settings. This can be observed in Figure ~\ref{fig:sub2} and Figure ~\ref{fig:sub4}. For both Fat-Tree and CORONET CONUS topologies, the throughput achieved by TMA is almost equal to the Upper Bound throughput achieved under ideal conditions. However, there is a slight deviation from the upper bound when 300 flows are routed through the Fat Tree topology. This is because of the resource limitations in the network. The ideal case assumes that the network is free of such limitations.

In the case of PDA, the parameter $\epsilon$ provides a trade-off between maximizing the network throughput and meeting all the flows' QoS constraints. The smaller the value of $\epsilon$, the closer the algorithm is to the offline scenario and the better the throughput will be. We can observe this in Figure ~\ref{fig:sub1} and Figure ~\ref{fig:sub3}. For both the topologies PDA($\epsilon$=0.5) can be seen giving consistently better throughput than PDA($\epsilon$=0.75). 

We can also observe from the results in Figure ~\ref{fig:sub1} and Figure ~\ref{fig:sub3} that PDA, irrespective of the value of $\epsilon$ is performing better than RA-RA. The throughput of RA-RA drops slightly at high number of flows. We believe that because RA-RA uses much simpler constraints when routing a flows and checks limited number of paths out of all available paths, it leads to less throughout. Even though, we have not bench marked the execution time of the three algorithms, we observed that RA-RA produces the results much faster than the PDA and TMA algorithms. 
