\section{Conclusion and Future Work}
\label{sec:conclusion}

In this paper, we did a comparative study of offline and online algorithms for dynamic service function chaining in SDN-Enabled Networks. We implemented a part of TMA algorithm and Primal-Dual-Update Algorithm proposed in \cite{ref:paper1} and RA-RA Algorithm proposed in \cite{ref:paper2} with the goal of maximizing the throughput. We observe that even if we consider offline scenario, the goal of achieving optimal throughput, given the constraints on resources is an NP-hard problem. Nevertheless, a convex relaxation of constrains in the practical setting can give us an optimal throughput. In the online setting, we compared the performance of PDA and RA-RA on two topologies, Fat-Tree and CORONET CONUS. We observed that PDA consistently gave better throughput than RA-RA even when the trade-off parameter, $\epsilon$, in the PDA algorithm was set to be more sensitive to meeting QoS requirements and allow less traffic through the network.

In the future, the comparison can be stretched in many directions. As discussed in the paper, for small flows we can use the offline algorithm (TFA) and for elephant flows, we can use the online (PDA) routing algorithm and compare the results. We can also conduct comparisons with various other topologies. 
Another aspect of Service function chaining is service function placement. As these service functions or network functions run on virtualized VM hardware, we can deploy the network functions anywhere in the network. Once we have deployed the network functions optimally by learning about the flows in the offline mode, we can use the offline routing algorithm and maximize the throughput. We can compare these results with our current simulation with fixed randomly distributed service functions.

